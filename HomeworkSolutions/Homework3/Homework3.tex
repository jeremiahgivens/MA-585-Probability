\documentclass[10pt,a4paper]{article}
\usepackage[utf8]{inputenc}
\usepackage[a4paper,%
            left=.75in,right=.75in,top=1in,bottom=1in]{geometry}
\setlength{\headsep}{0.25in}

\usepackage{amsthm}

\usepackage{graphicx}
\usepackage{pgfplots}

\usepackage{hyperref}

\hypersetup{colorlinks=true, linkcolor=blue, urlcolor=cyan}
            
\usepackage[english]{babel}

\theoremstyle{theorem}
\newtheorem{theorem}{Theorem}
\newtheorem{lemma}{Lemma}
\newtheorem{corollary}{Corollary}
\newtheorem{case}{Case}

\newcommand\restr[2]{{% we make the whole thing an ordinary symbol
  \left.\kern-\nulldelimiterspace % automatically resize the bar with \right
  #1 % the function
  \vphantom{\big|} % pretend it's a little taller at normal size
  \right|_{#2} % this is the delimiter
  }}

\theoremstyle{definition}
\newtheorem{definition}{Definition}
\newtheorem{remark}{Remark}

\usepackage{mathtools}
\DeclarePairedDelimiter\bra{\langle}{\rvert}
\DeclarePairedDelimiter\ket{\lvert}{\rangle}
\DeclarePairedDelimiterX\braket[2]{\langle}{\rangle}{#1 \delimsize\vert #2}

\usepackage{amsmath}
\usepackage{amsfonts}
\usepackage{amssymb}
\usepackage{fancyhdr}

\DeclareMathOperator{\interior}{int}

\newcommand{\Tau}{\mathcal{T}}

\newcommand{\Prob}{\mathbb{P}}

\newenvironment{amatrix}[1]{%
  \left(\begin{array}{@{}*{#1}{c}|c@{}}
}{%
  \end{array}\right)
}

\usepackage{calligra}
\DeclareMathAlphabet{\mathcalligra}{T1}{calligra}{m}{n}
\DeclareFontShape{T1}{calligra}{m}{n}{<->s*[2.2]callig15}{}

\newcommand{\scripty}[1]{\ensuremath{\mathcalligra{#1}}}

\pagestyle{fancy}
\author{Jeremiah Givens}
\newcommand{\subject}{Probability MA 585}
\newcommand{\Date}{9/2/2021} 
\makeatletter
\rhead{{\small\@author}}
\lhead{{\small\subject}}
\chead{{\large Homework 2}}
\cfoot{}
\rfoot{\thepage}
\lfoot{\today}

\renewcommand{\theequation}{\arabic{equation}}

\begin{document}
\section*{Problem 1.1}
Show that if $X \in C(0,1)$, then so is $1/X$.

\subsection*{Solution}
Since $X \in C(0, 1)$, we have 
\begin{align*}
f_X(x)  = \frac{1}{\pi} \cdot \frac{1}{1 + x^2} \text{,  for } - \infty < x < \infty.
\end{align*}
Define $Y = \frac{1}{X}$. Then, $X = \frac{1}{Y}$, and we have
\begin{align*}
|\det(J)| = \left|\frac{dx}{dy} \right| = \frac{1}{y^2}.
\end{align*}
Thus,
\begin{align*}
f_Y(y) &= |\det(J)|  f_X(1/y)\\
&= \frac{1}{y^2} \cdot \frac{1}{\pi} \cdot \frac{1}{1 + (1/y)^2}\\
&= \frac{1}{\pi} \cdot \frac{1}{1 + y^2},
\end{align*}
and we have shown that $Y \in C(0, 1)$.

\section*{Problem 1.8}
Show that if $X, Y$ are independent $N(0, 1)$-distributed random variables,  then $X/Y \in C(0, 1)$.

\subsection*{Solution}
We have
\begin{align*}
f_{X, Y} (x, y) &= \frac{1}{2 \pi} e^{-x^2 / 2} e^{-y^2 / 2}\\
&= \frac{1}{2 \pi} e^{-\frac{x^2 + y^2}{2}}.
\end{align*}
Define $U = X/Y$. We introduce the auxiliary variable $V = X$. Then we have the inverse relations
\begin{align*}
X &= V\\
Y &= V/U.
\end{align*}
With this, we can find our Jacobian determinant:
\begin{align*}
\det(J) &= 
\begin{vmatrix}
\frac{\partial x}{\partial u} & \frac{\partial x}{\partial v}\\
\frac{\partial y}{\partial u} & \frac{\partial y}{\partial v}
\end{vmatrix}\\
&= \begin{vmatrix}
0 & 1\\
\frac{-v}{u^2} & \frac{1}{u}
\end{vmatrix}\\
&= \frac{v}{u^2}.
\end{align*}
Thus,
\begin{align*}
f_{U, V} (u, v) &= |\det(J)| f_{X, Y} (v, v/u)\\
&= \frac{|v|}{u^2} \cdot  \frac{1}{2 \pi} e^{-\frac{v^2 + (v/u)^2}{2}}\\
&= \frac{|v|}{u^2} \cdot  \frac{1}{2 \pi} e^{-v^2 \frac{1 + (1/u)^2}{2}}.
\end{align*}
We want the distribution of $U$, so
\begin{align*}
f_U(u) &= \int_{- \infty}^\infty f_{U, V} (u, v) dv\\
&= \int_{- \infty}^\infty \frac{|v|}{u^2} \cdot  \frac{1}{2 \pi} e^{-v^2 \frac{1 + (1/u)^2}{2}} dv\\
&=  \frac{1}{u^2 \pi} \int_{0}^\infty v\cdot  e^{-v^2 \frac{1 + (1/u)^2}{2}} dv\\
&= \frac{1}{\pi} \cdot \frac{1}{1 + u^2}.
\end{align*}
Thus, we have shown that $U \in C(0, 1)$.

\section*{Problem 1.11}
Show that if $X$ and $Y$ are independent Exp($a$)-distributed random variables, then $X/Y \in F(2, 2)$.

\subsection*{Solution}
We have
\begin{align*}
f_{X, Y} (x, y) &= \frac{1}{a^2} e^{-x/a} e^{-y/a}\\
&= \frac{1}{a^2} e^{-\frac{x + y}{a}},
\end{align*}
for $0 \leq x, y < \infty$. Define $U$ and $V$ as we did in the previous problem. Then,
\begin{align*}
f_{U, V} (u, v) &= |\det(J)| f_{X, Y} (v, v/u)\\
&= \frac{v}{u^2} \cdot \frac{1}{a^2} e^{-\frac{v + v/u}{a}}\\
&=  \frac{v}{u^2} \cdot \frac{1}{a^2} e^{-v\frac{1 + 1/u}{a}}.
\end{align*}
Then, we can integrate to find the distribution of $U$:
\begin{align*}
f_U(u) &= \int_{- \infty}^\infty f_{U, V} (u, v) dv\\
&= \frac{1}{a^2 u^2} \int_{0}^\infty v e^{-v\frac{1 + 1/u}{a}} dv\\
&= \frac{1}{(u + 1)^2}.
\end{align*}
After looking up Fisher's distribution, we can conclude that $X/Y \in F(2, 2)$.


\section*{Problem 1.12}
Let $X, Y$ be independent random variables such that $X \in U(0, 1)$ and $Y \in U(0, \alpha)$. Find the density function of $Z = X + Y$.

\subsection*{Solution}
This is a test

\end{document}