\documentclass[10pt,a4paper]{article}
\usepackage[utf8]{inputenc}
\usepackage[a4paper,%
            left=.75in,right=.75in,top=1in,bottom=1in]{geometry}
\setlength{\headsep}{0.25in}

\usepackage{amsthm}

\usepackage{graphicx}
\usepackage{pgfplots}
            
\usepackage[english]{babel}

\theoremstyle{theorem}
\newtheorem{theorem}{Theorem}
\newtheorem{lemma}{Lemma}
\newtheorem{corollary}{Corollary}
\newtheorem{case}{Case}

\newcommand\restr[2]{{% we make the whole thing an ordinary symbol
  \left.\kern-\nulldelimiterspace % automatically resize the bar with \right
  #1 % the function
  \vphantom{\big|} % pretend it's a little taller at normal size
  \right|_{#2} % this is the delimiter
  }}

\theoremstyle{definition}
\newtheorem{definition}{Definition}
\newtheorem{remark}{Remark}

\usepackage{mathtools}
\DeclarePairedDelimiter\bra{\langle}{\rvert}
\DeclarePairedDelimiter\ket{\lvert}{\rangle}
\DeclarePairedDelimiterX\braket[2]{\langle}{\rangle}{#1 \delimsize\vert #2}

\usepackage{amsmath}
\usepackage{amsfonts}
\usepackage{amssymb}
\usepackage{fancyhdr}

\DeclareMathOperator{\interior}{int}

\newcommand{\Tau}{\mathcal{T}}

\newenvironment{amatrix}[1]{%
  \left(\begin{array}{@{}*{#1}{c}|c@{}}
}{%
  \end{array}\right)
}

\usepackage{calligra}
\DeclareMathAlphabet{\mathcalligra}{T1}{calligra}{m}{n}
\DeclareFontShape{T1}{calligra}{m}{n}{<->s*[2.2]callig15}{}

\newcommand{\scripty}[1]{\ensuremath{\mathcalligra{#1}}}

\pagestyle{fancy}
\author{Jeremiah Givens}
\newcommand{\subject}{Probability MA 585}
\newcommand{\Date}{9/2/2021} 
\makeatletter
\rhead{{\small\@author}}
\lhead{{\small\subject}}
\chead{{\large Homework 1}}
\cfoot{}
\rfoot{\thepage}
\lfoot{\today}

\renewcommand{\theequation}{\arabic{equation}}

\begin{document}
\section*{Problem 1}
Suppose one is rolling a six-sided die. Let $\mathcal{F}$ the smallest $\sigma$-field containing $A = \{5, 6\}$
and $B = \{2, 4, 6\}$.  Prove that $\{5\}$,  $\{1, 3\}$ and $\{1, 2, 3, 4\}$ are events for the measurable space
$(\Omega,\mathcal{F})$.

\subsection*{Solution}

\begin{proof}
Since $\mathcal{F}$ is a $\sigma$-algebra, we have that $\mathcal{F}$ is closed under complements and countable unions (and countable intersections, by D'Morgan's Law). In addition, $\mathcal{F}$ contains $\Omega = \{1, ...,6\}$, and the empty set $\emptyset$. Thus, if we can show that each of the three sets in the problem can be written as a finite union and/or intersection of $A, B$ and their complements, then our proof will be complete.

We have 
\begin{align*}
B^c \cap (A \cup B) &= \{1, 3, 5\} \cap \{2, 4, 5, 6\}\\
&= \{5\},
\end{align*}
and
\begin{align*}
B^c \cap A^c &= \{1, 3, 5\} \cap \{1, ..., 4\}\\
\{1, 3\},
\end{align*}
and
\begin{align*}
A^c = \{1, 2, 3, 4\}.
\end{align*}
With this, our proof is complete.
\end{proof}

\section*{Problem 2}
Let $A$ and $B$ be a pair of independent events. Prove that $A^c$ and $B^c$ are independent.

\subsection*{Solution}
\begin{proof}
We can see
\begin{align*}
P(A^c \cap B^c) &= P((A \cup B)^c) && \text{D'Morgan's Law}\\
&= 1 - P(A \cup B)\\
&= 1 - (P(A) + P(B) - P(A \cap B)) && \text{We derived this in class}\\
&= 1 - (P(A) + P(B) - P(A)P(B)) && \text{By independence of } A,B\\
&= 1 - (P(A) + P(B)(1-P(A))\\
&= 1 - (P(A) + P(B)P(A^c))\\
&= 1 - P(A) - P(B)P(A^c)\\
&= P(A^c) - P(B)P(A^c)\\
&= P(A^c)(1  - P(B))\\
&= P(A^c)P(B^c),
\end{align*}
and we can conclude that $A^c$ and $B^c$ are independent.
\end{proof}

\section*{Problem 3}
Consider a coin-die experiment: One flips a fair coin at first. If he gets a head, then he will
roll a 6-sided fair die; otherwise, he will roll a 4-sided unfair die, which has probability $\frac{5 - i}
{10}$ to
get the $i^\text{th}$ face up, where $i \in \{1,...,4\}$.  If one gets a 2 face up, what is the probability that they got a tail
when they flipped the coin?

\subsection*{Solution}
Let us denote the result of the coin flip as $X$.  Then, either $X = H$ or $X = T$ for a given experiment. Let us denote the result of the die roll by the random variable $Y$, defined in the obvious way. Let $A = \{X = T\}$ and $B = \{Y = 2\}$ be two events. By definition of conditional probability, we have
\begin{align*}
P(A|B) = \frac{P(A \cup B)}{P(B)} \text{, and } P(B|A) = \frac{P(A \cup B)}{P(A)} \implies P(A|B) = \frac{P(B|A) P(A)}{P(B)}.
\end{align*}
Now, since $\{X =H\}$ and $\{X = T\}$ partitions our sample space, we have that 
\begin{align*}
P(B) = P(B \cap \{X = H\}) + P(B \cap \{X = T\}) =  P(B| X = H) P(X = H) + P(B| X = T) P(X = T).
\end{align*}

Thus, the equation of interest becomes
\begin{align}
P(A|B) = \frac{P(B|A) P(A)}{P(B| X = H) P(X = H) + P(B| X = T) P(X = T)}.
\end{align}
The probabilities involved are all essentially given in the problem statement:
\begin{align*}
&P(A) = \frac{1}{2} = P(X = T) = P(X = H) && \text{because we flip a fair coin}\\
&P(B|X = H) = P(Y = 2| X = H) = \frac{1}{6} && \text{because we roll a fair die}\\
&P(B|A) = P(B|X = T) = P(Y = 2| X = T) = \frac{5 - 2}{10} = \frac{3}{10} && \text{because flip our weighted die}
\end{align*}
Plugging these into (1), we have
\begin{align*}
P(A|B) &= \frac{\frac{3}{10} \frac{1}{2}}{\frac{1}{6}  \frac{1}{2} + \frac{3}{10} \frac{1}{2}}\\
&= \frac{\frac{3}{20}}{\frac{1}{12} + \frac{3}{20}}\\
&= \frac{3}{\frac{5 + 9}{3}}\\
&= \frac{9}{14}.
\end{align*}
For a sanity check, the code for our numerical simulation can be found HERE, which supports our conclusion.
\end{document}