\documentclass[10pt,a4paper]{article}
\usepackage[utf8]{inputenc}
\usepackage[a4paper,%
            left=.75in,right=.75in,top=1in,bottom=1in]{geometry}
\setlength{\headsep}{0.25in}

\usepackage{amsthm}

\usepackage{graphicx}
\usepackage{pgfplots}

\usepackage{hyperref}

\hypersetup{colorlinks=true, linkcolor=blue, urlcolor=cyan}
            
\usepackage[english]{babel}

\theoremstyle{theorem}
\newtheorem{theorem}{Theorem}
\newtheorem{lemma}{Lemma}
\newtheorem{corollary}{Corollary}
\newtheorem{case}{Case}

\newcommand\restr[2]{{% we make the whole thing an ordinary symbol
  \left.\kern-\nulldelimiterspace % automatically resize the bar with \right
  #1 % the function
  \vphantom{\big|} % pretend it's a little taller at normal size
  \right|_{#2} % this is the delimiter
  }}

\theoremstyle{definition}
\newtheorem{definition}{Definition}
\newtheorem{remark}{Remark}

\usepackage{mathtools}
\DeclarePairedDelimiter\bra{\langle}{\rvert}
\DeclarePairedDelimiter\ket{\lvert}{\rangle}
\DeclarePairedDelimiterX\braket[2]{\langle}{\rangle}{#1 \delimsize\vert #2}

\usepackage{amsmath}
\usepackage{amsfonts}
\usepackage{amssymb}
\usepackage{fancyhdr}

\DeclareMathOperator{\interior}{int}

\newcommand{\Tau}{\mathcal{T}}

\newcommand{\Prob}{\mathbb{P}}

\newenvironment{amatrix}[1]{%
  \left(\begin{array}{@{}*{#1}{c}|c@{}}
}{%
  \end{array}\right)
}

\usepackage{calligra}
\DeclareMathAlphabet{\mathcalligra}{T1}{calligra}{m}{n}
\DeclareFontShape{T1}{calligra}{m}{n}{<->s*[2.2]callig15}{}

\newcommand{\scripty}[1]{\ensuremath{\mathcalligra{#1}}}

\pagestyle{fancy}
\author{Jeremiah Givens}
\newcommand{\subject}{Probability MA 585}
\newcommand{\Date}{9/2/2021} 
\makeatletter
\rhead{{\small\@author}}
\lhead{{\small\subject}}
\chead{{\large Homework 5}}
\cfoot{}
\rfoot{\thepage}
\lfoot{\today}

\renewcommand{\theequation}{\arabic{equation}}

\begin{document}
\section*{Problem 3.1}
The nonnegative, integer-valued, random variable $X$ has generating function $g_X(t) = \ln(\frac{1}{1 - qt})$. Determine $P(X = k)$ for $k = 0, 1, 2,...$, and determine $EX$ and Var$X$.

\subsection*{Solution}
By definition of a generating function, we have
\begin{align*}
g_X(t) &= Et^X\\
&= \sum_{k=0}^\infty P(X = k) t^k.
\end{align*}
As we derived in class, we have
\begin{align*}
P(X=k) &= \frac{g_{X}^{(k)}(0)}{k!},
\end{align*}
and
\begin{align}
EX(X-1)...(X - k + 1) = g_{X}^{(k)}(1).
\end{align}

Thus, we should find the derivatives of $g$:
\begin{align*}
g_{X}^{(1)}(t) &= \frac{q}{1 - qt}\\
g_{X}^{(2)}(t) &= \frac{q^2}{(1 - qt)^2}\\
g_{X}^{(3)}(t) &= \frac{2q^3}{(1 - qt)^3}\\
g_{X}^{(4)}(t) &= \frac{(3)(2)q^4}{(1 - qt)^4}\\
&.\\
&.\\
&.\\
g_{X}^{(k)}(t) &= \frac{(k-1)! \cdot q^k}{(1 - qt)^k}.
\end{align*}
Thus, we have
\begin{align*}
g_{X}^{(k)}(0) &= (k-1)! \cdot q^k
\end{align*}
for $k \in \mathbb{Z}^+$, and $g_X(0) = 0$.
We also have
\begin{align*}
g_{X}^{(k)}(1) &= \frac{(k-1)! \cdot q^k}{(1 - q)^k}
\end{align*}
for $k \in \mathbb{Z}^+$. With this, we have
\begin{align*}
P(X=k) &= \frac{g_{X}^{(k)}(0)}{k!}\\
&=  \frac{(k-1)! \cdot q^k}{k!}\\
&= \frac{q^k}{k}.
\end{align*}
Now, normalization allows us to solve for $q$:
\begin{align*}
1 &= \sum_{k=1}^\infty P(X = k)\\
&= \sum_{k=1}^\infty \frac{q^k}{k}\\
&= -\ln(1 - q) \text{ (Pulled from my big book of series)}\\
-1 &= \ln(1 - q)\\
e^{-1} &= 1 - q\\
q &= 1 - e^{-1}.
\end{align*}
Thus, we have that 
\begin{align*}
P(X = k) &= \boxed{\frac{(1 - e^{-1})^k}{k}}.
\end{align*}
Now, utilizing equation (1),
\begin{align*}
EX &= g_{X}^{(1)}(1)\\
&= \frac{q}{1 - q}\\
&= \boxed{\frac{1 - e^{-1}}{e^{-1}}}\\
EX(X-1) &= EX^2 - EX\\
&= g_X^{(2)}\\
&= \frac{(1 - e^{-1})^2}{e^{-2}}.
\end{align*}
Thus,
\begin{align*}
\text{Var}X &= EX^2 - (EX)^2\\
&= g_X^{(2)} + EX - (EX)^2\\
&= \frac{(1 - e^{-1})^2}{e^{-2}} + \frac{1 - e^{-1}}{e^{-1}} - \left( \frac{1 - e^{-1}}{e^{-1}} \right)^2\\
&=\boxed{\frac{1 - e^{-1}}{e^{-1}}}.
\end{align*}

\section*{Problem 3.4}
Suppose that $Y$ is a random variable such that 
\begin{align*}
EY^k = \frac{1}{4} + 2^{k-1} \text{, for } k=1,2,....
\end{align*}
Determine the distribution of $Y$.

\subsection*{Solution}

\end{document}